\documentclass[10pt,]{article}
\usepackage[compact]{titlesec}
\usepackage[T1]{fontenc}
\usepackage{lmodern}
\usepackage{amssymb,amsmath}
\usepackage{ifxetex,ifluatex}
\usepackage{fixltx2e} % provides \textsubscript
\usepackage[ampersand]{easylist}
\usepackage[export,Export]{adjustbox}
% use microtype if available
\IfFileExists{microtype.sty}{\usepackage{microtype}}{}
% use upquote if available, for straight quotes in verbatim environments
\IfFileExists{upquote.sty}{\usepackage{upquote}}{}
\ifnum 0\ifxetex 1\fi\ifluatex 1\fi=0 % if pdftex
  \usepackage[utf8]{inputenc}
\else % if luatex or xelatex
  \usepackage{fontspec}
  \ifxetex
    \usepackage{xltxtra,xunicode}
  \fi
  \defaultfontfeatures{Mapping=tex-text,Scale=MatchLowercase}
  \newcommand{\euro}{€}
\fi
\usepackage[a4paper,margin=20mm]{geometry}
\usepackage{longtable}
\ifxetex
  \usepackage[setpagesize=false, % page size defined by xetex
              unicode=false, % unicode breaks when used with xetex
              xetex]{hyperref}
\else
  \usepackage[unicode=true]{hyperref}
\fi
\hypersetup{breaklinks=true,
            bookmarks=true,
            pdfauthor={},
            pdftitle={},
            colorlinks=true,
            urlcolor=blue,
            linkcolor=magenta,
            pdfborder={0 0 0}}
\urlstyle{same}  % don't use monospace font for urls
\setlength{\parindent}{0pt}
\setlength{\parskip}{6pt plus 2pt minus 1pt}
\setlength{\emergencystretch}{3em}  % prevent overfull lines
\setcounter{secnumdepth}{0}
% vim:ft=tex

\usepackage{tabu}
\usepackage{soul}
\usepackage{lmodern}
\renewcommand*\familydefault{\sfdefault}
\usepackage[T1]{fontenc}
\usepackage{fix-cm}
\usepackage{calc}
\usepackage{graphicx}
\usepackage[dvipsnames,usenames]{color}
\usepackage{tikz}
\usepackage[all,top]{background}
    \SetBgContents{
       \begin{tikzpicture}[remember picture,overlay]
          \shade[top color=blue!25!white,middle color=white, bottom color=white]
          (-0.5\paperwidth,1ex) rectangle (\paperwidth,-0.05\paperheight) ;
          \shade[bottom color=black!25!white,middle color=white,top color=white]
          (-0.5\paperwidth,-0.95\paperheight) rectangle
          (\paperwidth,-\paperheight) ;
        \end{tikzpicture}
    }
    \SetBgOpacity{1}
    \SetBgScale{1}
    \SetBgAngle{0}
\usepackage{wrapfig}
\usepackage{varwidth}

\author{}
\date{}

\begin{document}

\newpage
\null
\vfill

\begin{minipage}[c]{\columnwidth}
    \centering
    \fontsize{25}{2ex}\selectfont\bfseries COMBINE Student Symposium 2013
\end{minipage}

\begin{longtable}[c]{@{}ll@{}}
\hline\noalign{\medskip}

\noalign{\medskip}
\begin{minipage}[t]{0.30\columnwidth}\raggedright
    \emph{Date \& Time}
    \end{minipage} & \begin{minipage}[t]{0.70\columnwidth}\raggedright
    Thursday, 28th November, 2013\\
    11:00am to 4:45pm\\
    (Join us for drinks at the conclusion of the symposium!)
    \end{minipage}
\\\noalign{\medskip}
\begin{minipage}[t]{0.30\columnwidth}\raggedright
\emph{Venue}
\end{minipage} & \begin{minipage}[t]{0.70\columnwidth}\raggedright
    Denis Driscoll Theatrette\\
    Level 3, Doug McDonell Building\\
    The University of Melbourne\\[1em]
    Google maps: \href{http://goo.gl/maps/rpGnM}{http://goo.gl/maps/rpGnM}\\
    Campus ref: J19 (Building 168)
\end{minipage}
\\\noalign{\medskip}
\begin{minipage}[t]{0.30\columnwidth}\raggedright
\emph{Registration/RSVP}
\end{minipage} & \begin{minipage}[t]{0.70\columnwidth}\raggedright
\href{http://goo.gl/n7OASr}{http://goo.gl/n7OASr}, registration is
\emph{FREE}.
\end{minipage}
\\\noalign{\medskip}
%   \begin{minipage}[t]{0.30\columnwidth}\raggedright
%   \emph{Abstract Submission URL}
%   \end{minipage} & \begin{minipage}[t]{0.70\columnwidth}\raggedright
%       \href{http://goo.gl/gBb9yA}{http://goo.gl/gBb9yA}
%   \end{minipage}
%   \\\noalign{\medskip}

%   \begin{minipage}[t]{0.30\columnwidth}\raggedright
%   \emph{Submission Deadline}
%   \end{minipage} & \begin{minipage}[t]{0.70\columnwidth}\raggedright
%       \st{1st November, 2013} \bfseries 10th November, 2013 (deadline extended)
%   \end{minipage}
%   \\\noalign{\medskip}

%   \begin{minipage}[t]{0.30\columnwidth}\raggedright
%   \emph{Author Notification}
%   \end{minipage} & \begin{minipage}[t]{0.70\columnwidth}\raggedright
%       15th November, 2013
%   \end{minipage}
%   \\\noalign{\medskip}

\begin{minipage}[t]{0.30\columnwidth}\raggedright
\emph{Enquiries}
\end{minipage} & \begin{minipage}[t]{0.70\columnwidth}\raggedright
\href{mailto:symposium@combine.org.au}{symposium@combine.org.au}
\end{minipage}
\\\noalign{\medskip}
\hline
\end{longtable}

%   \subsubsection{Submission Guidelines}

%       Abstracts are to contain no more than 200 words and must
%       include a brief description of your work, any major findings,
%       and its impact/significance to the field.
%       You can find tips on how to write a good abstract for a presentation at: \href{http://www.ncbi.nlm.nih.gov/pmc/articles/PMC3136027/}{http://www.ncbi.nlm.nih.gov/pmc/articles/PMC3136027/}

\subsubsection{Description}

The COMBINE Student Symposium 2013 is an opportunity for students and
early-career researchers in computational biology, bioinformatics, and the life
sciences to share their research with others in these fields. Presentations
will be judged and the best ones will receive prizes!

The symposium is not a formal peer-reviewed conference, but a less formal
opportunity for participants to refine presentations and obtain feedback on
their research prior to major upcoming conferences.

Presentation topics may include, but are not limited to:\par

\begin{minipage}[c]{\linewidth}
    \centering
    \begin{varwidth}[c]{\linewidth}
        \raggedright
        \bfseries
        sequence analysis,evolution and phylogeny, comparative genomics, protein
        structure, molecular and supramolecular dynamics, molecular evolution, gene
        regulation and transcriptomics, RNA biology, proteomics, systems biology,
        statistical genetics, mathematical biology \normalfont\ldots and much more.
        \end{varwidth}
\end{minipage}\par\bigskip


% Copy-paste this higher into the document if you want to draw more attention (i guess).

\subsubsection{Careers Panel}

At the conclusion of the oral presentations, we will be hosting a Q\&A
session with a panel of established researchers
relating to career and professional-development, as well as to share
advice and lessons learnt from their own experience in research.

\begin{minipage}[c]{\linewidth}
    \centering
    Our Panelists:\\[3ex]
    \begin{tabu}{X[c]X[c]X[c]}
        \textbf{Prof.~Christopher Leckie}\linebreak
            Dept.~of Computing and Info.~Systems\linebreak
            The University of Melbourne
        & \textbf{Dr.~Annalisa Swan}\linebreak
            Researcher\linebreak
            IBM Melbourne Research Laboratory
        & \textbf{Dr.~Tony Papenfuss}\linebreak
            Laboratory Head (Bioinformatics)\linebreak
            The Walter and Eliza Hall Institute of Medical Research (WEHI)
    \end{tabu}
    \vspace{3ex}
    \begin{tabu}{X[c]X[c]}
        \textbf{Dr.~Armita Zarnegar}\linebreak
            Computational Biologist\linebreak
            Victorian Department of Primary Industries
        & \textbf{Dr.~Michael J.~Kuiper}\linebreak
            Computational Molecular Scientist\linebreak
            The Victorian Life Sciences Computation Initiative (VLSCI)
    \end{tabu}
\end{minipage}

%%%%%%%%%%%%%%%%%%%%%%%%%%%%%%%%%%%%%%%%%%%%%%%%%%%%%%%%%%%%%%%%%%%%%%%%%%%%%%%%%%%%%%%%%

\subsubsection{About Us}

\href{http://www.combine.org.au}{COMBINE} is the official
\href{http://www.iscbsc.org/content/regional-student-groups}{ISCB
Regional Student Group} for Australia, a student-run organisation for
researchers in computational biology, bioinformatics, and related
fields.

For more information about COMBINE, upcoming events, or to become a
volunteer, please visit our website at
\href{http://www.combine.org.au}{combine.org.au}.

We are proudly sponsored by:

 \vspace{2ex}
\begin{minipage}[c]{0.5\columnwidth}
    \centering
    \includegraphics[height=15mm]{./images/ICT-for-Life-Sciences-Forum-logo.png}\quad
    \includegraphics[height=15mm]{./images/ISCBSC-logo.png}
\end{minipage}
\begin{minipage}[c]{0.5\columnwidth}
\begin{itemize}
    \itemsep1pt\parskip0pt\parsep0pt
    \item
      \href{http://www.ict4lifesciences.org.au}{ICT for Life Sciences Forum}
    \item
      \href{http://www.iscb.org}{International Society for Computational
      Biology (ISCB)}
    \end{itemize}
\end{minipage}


\vfill
\pagebreak
\null
\vfill
\section{Symposium Program}

\centering

\begin{minipage}[c]{0.8\linewidth}
    \begin{itemize}
        \item[11:00am] Registration \& Morning Tea
        \item[11:30am] Welcome
        \item[11:40am] Session 1
            \begin{itemize}
                \item \emph{\bfseries Integrated genomic analysis of structural rearrangements in hormone-driven cancers}\\Speaker: Marek Cmero
                \item \emph{\bfseries Monogamy is in your genes: Genotyping microsatellites in next-generation sequencing data}\\Speaker: Harriet Dashnow
                \item \emph{\bfseries Investigating fragmentation signatures of cell-free DNA in human plasma using next-generation sequencing}\\Speaker: Dineika Chandrananda
                \item \emph{\bfseries The Plasmodium paradox: Generalisation with specialisation}\\Speaker: Edward Tasker
            \end{itemize}
        \item[1:00pm] Lunch
        \item[2:00pm] Session 2
            \begin{itemize}
                \item \emph{\bfseries A whole kidney network model}\\Speaker: Tom Gale
                \item \emph{\bfseries Identification of novel therapeutics for complex diseases from genome wide association data}\\Speaker: Mani Grover
                \item \emph{\bfseries CoRSeqV3: A Novel Sequence Based Algorithm for Predicting HIV Coreceptor Usage}\\Speaker: Kieran Cashin
            \end{itemize}
        \item[3:00pm] Afternoon Tea
        \item[3:30pm] Careers Panel
        \item[4:30pm] Awards \& Symposium Close
        \item[5:00pm] Social Event (optional)
    \end{itemize}
\end{minipage}

\vspace{5ex}

\begin{minipage}[c]{0.8\linewidth}
    \centering
    The COMBINE Student Symposium 2013 is proudly sponsored by:\\[3ex]
    \begin{varwidth}[t]{0.45\linewidth}
        \centering
        \includegraphics[height=25mm,valign=M]{./images/logo_vlsci_3508x1890.jpg}\\[2ex]
        The Victorian Life Sciences Computation Initiative\\
        \href{http://www.vlsci.org.au}{www.vlsci.org.au}
    \end{varwidth}
    \qquad
    \begin{varwidth}[t]{0.45\linewidth}
        \centering
        \includegraphics[height=25mm,valign=M]{./images/ICT-for-Life-Sciences-Forum-logo.png}\\[2ex]
        ICT For Life Sciences forum\\
        \href{http://ict4lifesciences.org.au/}{www.ict4lifesciences.org.au}
    \end{varwidth}
    
\end{minipage}

\vfill
\pagebreak
\null
\vfill

\small

{\Large\bfseries\centering Oral Presentation Abstracts (Session 1)}\\[3ex]

\begin{minipage}[c]{\linewidth}
\raggedright

\emph{\bfseries Integrated genomic analysis of structural rearrangements in hormone-driven cancers}\\
Speaker: Marek Cmero (NICTA)\\[1ex]
{\small
Genomic translocations have the potential to disrupt gene function,
effectively driving the pathogenesis of cancers. Computational techniques used
for the detection of translocation events are often highly sensitive to noise
and therefore produce false-positive results. By integrating DNA and RNA data
from the same patient samples, we have substantially increased the specificity
of genomic translocations. Genomic breakpoints from Socrates, a soft-clip
based detector of rearrangements, was used to validate detected fusion
transcripts from deFuse, an RNA fusion detector, substantially reducing false
positive results. Additionally, we have profiled the DNA nucleotide content
proximal to breakpoint locations and have observed patterns of AT content and
DNase hypersensitivity surrounding the breakpoint point of origin from
hormone-driven cancers, distinct from the signatures observed in breaks within
non-hormone driven cancers. The interaction of hormone receptors with
chromatin may be responsible for the inducing translocations. We have
developed a convolution-based classifier incorporating AT content and DNase
hypersensitivity to identify hormone-driven translocations, and potentially
highlight genomic regions prone to breaking.}

\vspace{3ex}

\emph{\bfseries Monogamy is in your genes: Genotyping microsatellites in next-generation sequencing data}\\
Speaker: Harriet Dashnow (Murdoch Childrens Research Institute)\\[1ex]

{\small
There is a gene linked to monogamy. The first inklings of this connection came
from studies of prairie voles, where monogamy is triggered by repetitive DNA
sequences in a gene called arginine vasopressin receptor 1a (avpr1a). In
humans, repeats in the same gene are associated with monogamy, as well as age
of first sexual intercourse and marriage satisfaction (as you might expect!).
These same repeats in are associated with the impaired social interactions that
characterise Autism.  These repetitive DNA sequences are called
microsatellites. The human avpr1a gene contains four microsatellites, such as
ACACACAC. Given their links to social behaviour and disorder, measuring
microsatellites is vital. The problem is that repetitive DNA is hard to
measure. Existing lab-based methods are inaccurate, time-consuming, and
expensive.  Enter next-generation sequencing. Finally we can do DNA sequencing
on a massive scale, fast and cheap enough for large psychological studies. I
will compare how well Illumna MiSeq and IonTorrent can sequence
microsatellites, as well as doing lab-based validation.  I have developed a
method that leverages next-generation sequencing technologies to measure
complex microsatellites. No other current software tool can do this accurately.
I present the performance of my algorithm on ultra-deep sequencing data from
four microsatellites in the avpr1a gene.  These analysis methods and software
will help us understand how genes shape our relationships and reveal the link
between genetics, social behaviour and autism.}

\vspace{3ex}

\emph{\bfseries Investigating fragmentation signatures of cell-free DNA in human plasma using next-generation sequencing}\\
Speaker: Dineika Chandrananda (Walter and Eliza Hall Institute)\\[1ex]

{\small
Cell-free DNA can be found in many fluids of the human body. This DNA is
thought to escape from cells undergoing apoptosis or be actively secreted from
cells before entering fluids such as the plasma, urine and cerebrospinal fluid.
High-throughput studies have shown that cell-free DNA exists as short fragments
of somewhat predictable lengths, leading us to believe that there are
biological signatures present in the naturally cleaved fragments.

Using deep-sequencing data of cell-free DNA from the plasma of pregnant women,
we investigate how the circulating fragments localize to nucleosomal arrays on
a genome-wide level and how the fragment sizes are dependent on the genomic
location. We also demonstrate the existence of significant fragmentation bias
in cell-free DNA where nucleotide frequencies show a position specific pattern
in the region spanning 8-9 positions on either side of the DNA cleavage site.
This bias stemming from motif-specific cleavage is related to the GC-content in
genomic regions, which should be considered when correcting the GC bias in
cell-free DNA sequencing data. Lastly, since maternal plasma is a mixture of
fragments from the mother and fetus, we compare the aforementioned biological
signatures between maternal and fetal DNA to assess the differences.}

\vspace{3ex}

\emph{\bfseries The Plasmodium paradox: Generalisation with specialisation}\\
Speaker: Edward Tasker (Monash University)\\[1ex]

{\small
Parasites of the genus Plasmodium are most famous for causing human malaria. It
is less well known that Plasmodium species infect all three classes of land
vertebrates. Despite this broad-host spectrum, individual Plasmodium species
are generally highly host-specific. We refer to this specificity in the face of
diversity as the Plasmodium paradox. We hypothesise that host-specialisation is
driven by intense host-parasite interactions, in particular at parasite
virulence genes. To investigate this proposition, we are applying Bayesian
genomic segmentation methods to genome alignment of three Plasmodium lineages
infecting humans (P. falciparum), chimpanzee (P. reichenowi) and chicken (P.
gallinaceum). Our goal is to identify segments that are as divergent between
human and chimpanzee malarias as each is from chicken malaria (i.e., conserved
over large evolutionary distance, but variable over smaller distances), and
locate them in relation to described coding-regions. Our algorithm has
identified a model with 13 segment classes to represent this data. We are
currently analysing each class by mapping the segments back to the P.
falciparum genome. Our work will help identify genomic regions likely important
to the ability of malaria parasites to infect their hosts; and it will
contribute to our understanding of how host-parasite interactions shape
parasite genomes.}

\end{minipage}

\vfill
\pagebreak
\null
\vfill

{\Large\bfseries\centering Oral Presentation Abstracts (Session 2)}\\[3ex]

\begin{minipage}[c]{\linewidth}
\raggedright
\small

\emph{\bfseries A whole kidney network model}\\
Speaker: Tom Gale (The University of Melbourne)\\[1ex]
{ \small
Existing computational models of kidney physiology either describe the whole
organ using lumped parameters or only consider local function. This work aims
to integrate modelling at large and small scales together by producing a
network model of a whole rat kidney consisting of many communicating small
scale components, then validate that model by comparison with animal data and
other existing computational models.

We have developed algorithms to computer generate artery and nephron
structures, similar to those found in rat kidneys. This generated kidney
structure is then used for physiological simulation. The physiological
simulation uses a network arterial model based on compliance and Poiseuille
flow connected to an existing nephron model by Moss (2009), with some
modification to connect to the artery model.

Simulation of 16 nephrons reproduces results from the computer model by Marsh
(2013). Whole rat kidney simulations with 30,000 nephrons show expected
behaviour, such as blood flow rate and glomerular filtration rate. Comparison
with 50 nephron laser speckle and partial nephrectomy animal data is also
shown. These results from simulations at a range of different scales provide
evidence to demonstrate that this first whole rat kidney model composed of
individual nephron models is valid.}

\vspace{3ex}

\emph{\bfseries Identification of novel therapeutics for complex diseases from genome wide association data}\\
Speaker: Mani Grover (Deakin University)\\[1ex]
{ \small
Human genome sequencing has enabled the association of
phenotypes with genetic loci, but our ability to effectively translate this
data to the clinic has not kept pace. Integration of drug-target data with
candidate gene prediction systems can identify novel phenotypes which may
benefit from current therapeutics. Such a drug repositioning tool can save
valuable time and money spent on preclinical studies and phase I clinical
trials. We adopted a simple approach to integrate drug data with candidate
gene predictions at the systems level. We previously used Gentrepid as a
platform to predict 1,497 candidate genes for seven complex diseases considered
in the Wellcome Trust Case Control Consortium genome wide association study.
Using the publicly available drug databases, Therapeutic Target Database,
PharmGKB and DrugBank as sources of drug-target association data, we identified
a total of 428 (29\%) candidate genes as novel therapeutic targets for the
phenotype of interest and 2,130 drugs feasible for repositioning against the
predicted targets.  By integrating genetic, bioinformatic and drug data, we
have demonstrated that currently available drugs may be repositioned as novel
therapeutics for the seven diseases studied here, quickly taking advantage of
prior work in pharmaceutics to translate ground breaking results in genetics to
clinical treatments.}

\vspace{3ex}

\emph{\bfseries CoRSeqV3: A Novel Sequence Based Algorithm for Predicting HIV Coreceptor Usage}\\
Speaker: Kieran Cashin (Burnet Institute)\\[1ex]
{\small
Human immunodeficiency virus (HIV) gains entry into immune cells by binding one
of two cell-surface coreceptor molecules, either CCR5 or CXCR4. Maraviroc (MVC)
is an anti-HIV drug that binds to- and blocks- HIV from using CCR5. MVC is only
active against CCR5-using HIV strains and therefore is not prescribed to
patients harbouring CXCR4-using viruses. Unfortunately, the cost, long
turn-around time and technical difficulty of the laboratory experiment required
to determine patient HIV CCR5- or CXCR4-usage have limited prescription of MVC.
In response, we have developed an in silico algorithm (CoRSeqV3) that analyses
protein sequences translated from a small portion of HIV DNA to accurately
predict HIV CCR5- or CXCR4-usage. CoRSeqV3 prediction criteria were developed
by analysing thousands of published HIV protein sequences with known CCR5- or
CXCR4-usage in order to elucidate sequence mutations and characteristics that
differentiate CCR5- from CXCR4-using viruses. CoRSeqV3 is quick, cheap and can
be performed by most diagnostic labs worldwide. Importantly, we have shown that
CoRSeqV3 can reliably predict the coreceptor usage of all major HIV subtypes,
including those that predominate in India, China, Southern Africa and Southeast
Asia, namely regions of the world burdened most by the HIV pandemic.}

\end{minipage}

\vfill

\end{document}
